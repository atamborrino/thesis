\documentclass[a4paper,11pt]{kth-mag}
\usepackage[T1]{fontenc}
\usepackage{textcomp}
\usepackage{lmodern}
\usepackage[latin1]{inputenc}
\usepackage[swedish,english]{babel}
\usepackage{modifications}
\usepackage[backend=bibtex]{biblatex}
% code
\usepackage{listings}
\usepackage{color}

\usepackage{minted}

\usepackage{graphicx}

\usepackage{caption}

\newcommand{\codesize}{\footnotesize}

\pdfoptionpdfminorversion 6

\bibliography{bib/database.bib}

\definecolor{dkgreen}{rgb}{0,0.6,0}
\definecolor{gray}{rgb}{0.5,0.5,0.5}
\definecolor{mauve}{rgb}{0.58,0,0.82}

\lstset{frame=tb,
  language=Java,
  aboveskip=3mm,
  belowskip=3mm,
  showstringspaces=false,
  columns=flexible,
  basicstyle={\small\ttfamily},
  numbers=none,
  keywordstyle=\color{blue},
  commentstyle=\color{dkgreen},
  stringstyle=\color{mauve},
  breaklines=true,
  breakatwhitespace=true
  tabsize=3
}


\title{A Real-Time Reactive platform for Data Integration and Event Stream Processing}

% \subtitle{Duis autem vel eum iruire dolor in hendrerit in
%           vulputate velit esse molestie consequat, vel illum
%           dolore eu feugiat null}
% \foreigntitle{Lorem ipsum dolor sit amet, sed diam nonummy nibh eui
%               mod tincidunt ut laoreet dol}
\author{Alexandre Tamborrino}
\date{June 2014}
\blurb{Master's Thesis at Zengularity, examined by KTH ICT\\Supervisor: Antoine Michel\\Examiner: Vladimir Vlassov}
\trita{TRITA xxx yyyy-nn}
\begin{document}
\frontmatter
\pagestyle{empty}
\removepagenumbers
\maketitle
\selectlanguage{english}

\begin{abstract}
In this thesis is presented a Real-time Reactive platform for Data Integration and Event Stream Processing. The platform allows to define data pullers that incrementally pull data changes from REST data sources and propagates them as streams of immutable events across the system according to the Event-Sourcing principle. The Stream Processing part is a Tree-like structure of event-sourced stream processors where a processor can react in various ways to events sent by its parent and send derived sub-streams of events to child processors. A processor use case is maintaining a pre-computed view on aggregated data, which allows to define low read latency business dashboards that are updated in real-time.
The platform follows the Reactive architecture principles to maximize performance and minimize resource consumption using an asynchronous non-blocking architecture with an adaptive push-pull stream processing model with automatic back-pressure. Moreover, the platform uses functional programming abstractions for simple and composable asynchronous programming.

\end{abstract}
\clearpage

% \begin{foreignabstract}{swedish}
%   Denna fil ger ett avhandlingsskelett.
%   Mer information om \LaTeX-mallen finns i
%   dokumentationen till paketet.

% Lorem ipsum dolor sit amet, consectetuer adipiscing elit. Maurisd
% purus. Fusce tempor. Nulla facilisi. Sed at turpis. Phasellus eu
% ipsum. Nam porttitor laoreet nulla. Phasellus massa massa, auctor
% rutrum, vehicula ut, porttitor a, massa. Pellentesque fringilla. Duis
% nibh risus, venenatis ac, tempor sed, vestibulum at, tellus. Class
% aptent taciti sociosqu ad litora torquent per conubia nostra, per
% inceptos hymenaeos.
% \end{foreignabstract}
% \clearpage

\tableofcontents*

\mainmatter
\pagestyle{newchap}

\chapter{Introduction}

Data is now at the center of organizations and is increasingly heterogeneous, meaning that there is an explosion of data sources 
that each exposes data in its own format that can be structured, semi-structured or non-structured.
Another major trend is that data processing needs to be real-time, because managers no longer want to wait a whole day 
to have reports and alerts on their business data.
\\

To achieve these requirements, traditional monolithic Data Wharehouse softwares start to be out-dated. They often
propose to deal with only structured data to map it in a relational model, and are often batch-oriented: 
the ETL mechanism (data extraction, transform and load) regularly happen once or twice day, and there is no mechanism 
for real-time subscriptions of new events happening on the data as highlighted by Jay Kreps, principal staff engineer at LinkedIn \footfullcite{bib:linkedinLog}.
\\

The platform I present in this thesis is an event-oriented Data Integration and Stream Processing platform that allows data consumers
to subscribe to data changes in real-time. The main shift from traditional Data Wharehouse softwares is that the whole
system is organized around the notion of \textit{immutable event}. Such a data system captures from various data sources
a historical record of immutable events. Each event represents the change (creation, update or deletion) made to the data at a particular time. 
Based on the Event-Sourcing principle \footfullcite{bib:eventSourcing}, events are stored in a Journal that is an ordered sequence of 
events. Then, the stream of events coming from the Journal can be processed by data consumers that can react to the change of data. 
An example of data consumer can be one that maintain a pre-computed view on the data that is updated upon each event, or one that push 
notification to another service upon a certain kind of events (see Figure \ref{fig:main_archi} for the global architecture).
\\

An example use case is when an organization uses different SaaS services for each of its teams. For instance, the sales
team uses a SaaS software to process their sales pipeline, the project management team uses another SaaS software to manage
the production teams, etc... Without a central data backbone, it is not possible to have a global view on the company data.
The platform I present in this thesis can integrate these different SaaS softwares using their REST API, detect what
changes have been made on the data, and emits the corresponding events. As a result, data consumers can use these events
to update dashboards about the company data in real-time, mixing the data coming from different sources. A data consumer can also push a
notification to SaaS service X when it receives an event from SaaS service Y, allowing real-time synchronization between
heterogeneous services.
\\

An advantage of Event Sourcing is that the whole history of the system is stored. Events are immutable changes made to the data
and are always appended to the Journal (never deleted or modified). As a result, the system stores not only the current state
of the data, but also all its previous states. This allows two interesting properties.

First, it is possible to query past states of the data. This can be very useful for various use cases
where one is interested in the data history, for example a financial audit.

Moreover, storing all the data changes greatly improves the fault-tolerance of the system. As events are not deleted, it is always possible 
to come back in the past in the Journal, delete some delete events that were put by mistake, and replay the events after
them to re-build the system in a right state. This is also referred as Human Fault-Tolerance \footfullcite{bib:human-ft}: in a mutable system,
if an user accidentally delete a data entry, it is lost for ever. But in an immutable system, the deletion is just another
event added to the journal. Figure \ref{fig:event-sourcing} illustrates the difference between a mutable system and an immutable event-sourced system.
\\ 
\begin{figure}[h]
  \begin{center}
    \makebox[\textwidth]{\includegraphics[width=1.0\textwidth]{img/event-sourcing.png}}
    \caption{Immutable datastore and the Event Sourcing principle}
    \label{fig:event-sourcing}
  \end{center}
\end{figure}

This kind of architecture is also known as CQRS \footfullcite{bib:cqrs}. The core principle of CQRS is to decouple the write part and the read part
of a system. The write part (Data Integration) only needs to push immutable events to the Journal in an append-only fashion, which
is very efficient because there is no mutation of the data and no read-write contentions as in traditional databases.
The read part is a set of denormalized pre-computed view that are optimized for low read latency (as the view are automatically re-computed
when a new related event comes in).
An obvious downside of such an architecture is that data is eventually consistent: when a data producer has received the acknowledgment
from the Journal, there is no guarantee that data consumers has already processed the event and updated the data view. However eventual consistency is not
a problem for the functional requirements of the platform.

This model also allows very easy distribution of the platform. Easy distribution because it is a message-oriented
architecture where each component (data producer, journal, data consumers with data views) only exchanges messages (events) to each other (share-nothing architecture).
\\

The platform is composed of three main parts: 
\begin{itemize}
  \item Data Integration, that must integrate several data sources in order to emit 
events (data changes) to the Journal. 
  \item Journal, an abstraction for a sequence of immutable events. The Journal must expose methods to insert events,
  and expose methods to subscribe to the stream of events.
  \item Stream processing, where one can define a tree of data consumers (stream processors) that can react to
  events, maintain derived pre-computed view on the data, and emit new stream of events.
\end{itemize}

\begin{figure}[h]
  \begin{center}
    \makebox[\textwidth]{\includegraphics[width=1.0\textwidth]{img/main_archi.png}}
    \caption{Global architecture}
    \label{fig:main_archi}
  \end{center}
\end{figure}

Nonetheless, this kind of evented architecture must be done with a lot of care concerning technical architecture.
The platform needs to do lot of IO in order to push the stream of events from data sources to data consumers, and must
parallelize a lot of operations. Moreover, it must ensure that the stream of events (producers) does not overwhelm the stream
processors (consumers), i.e if consumers process data slowly, producers must try to slow the push rate. The platform should also deal with possible
failure of components and offer strong guarantees on these cases (like no message loss or duplication).

In order to fulfill those requirements, the platform will apply the principles of the Reactive Manifesto \footfullcite{bib:reactiveManifesto} in order to guarantee
that the platform is \textit{scalable, event-driven, resilient and responsive} (the four Reactive Traits). An asynchronous non-blocking approach with a share-nothing architecture will be used to develop the platform in order to optimize resource consumption, decouple components to be able to distribute them easily, take easily advantage of parallelization and handle failures. 
The platform is developed using functional programming in the Scala programming language \footfullcite{bib:scala} in order to leverage functional programming abstractions to better handle asynchronous and stream-oriented code.
\\


\chapter{Requirements}

\section{Functional requirements}

\subsection{Data Integration}

The Data Integration part of the platform needs to integrate several data sources to the Journal. Integration means that it must be able 
to detect the changes made to the data, and push events that can be either create event or update event or delete event.
In the following, we call a data entry a \textit{resource}. A resource is a keyed data defined by its id (for example \verb|/client/1| for a
resource of type client of id 1). Each type of resource has a defined set of fields (for example a client
will have a field name, address, ...).
\\

More specifically, the platform needs to integrate several data sources that expose REST APIs. Such APIs expose information
concerning business data such as new sales information, new financial information, new production information...
Each time that a resource is modified in one of these data sources, the platform should detect this change, apply some data cleaning and transformation, create an 
event from it and push it into the Journal.

The problem with most REST APIs is that they are not evented, i.e they are pulled-based and not pushed-based. 
One must sent an HTTP request to query new data each time they need to. There exists some techniques to stream data via HTTP 1.1 and the 
Chunked Transfer Encoding, but the REST APIs that the platform needs to integrate do not expose such stream interface. 
Thus, the architecture of this part needs to provide a way to perform incremental pull from data sources, and then transform
it in a push stream of events towards the Journal. Moreover, the platform needs to make sure to insert the events in the same
order that they happened in their data source.


\subsection{Journal and Stream Processing}

The Journal must provide a way for data producers to push one or several events that represent the creation, update or 
delete of a resource. Moreover, it must allow data consumers to subscribe to the stream of inserted events. Events must be
immutable and are stored in a sequence that respects the insertion order. The stream of events pushed to the data consumers
(stream processors) must be in the same order than the insertion order and with no event loss or duplication. Of course, the Journal must be persistent to
be able to recover its data after a shutdown or a crash.
\\

The Stream Processing part is the most complex part of the platform. This part should be a library that allows the user
to define a tree a stream processors (see Figure \ref{fig:tree}), where the root of the tree is the Journal. 

\begin{figure}[h]
  \begin{center} 
    \makebox[\textwidth]{\includegraphics[width=1.0\textwidth]{img/tree.png}}
    \caption{A tree of stream processors}
    \label{fig:tree}
  \end{center}
\end{figure}

A stream processor receives events coming from its parent node. Upon the receive of an event, it can do one or several of these
actions (see Figure \ref{fig:streamprocessor}):
\begin{itemize}
  \item Creation of a sub-stream: the stream processor can transform a received event to a stream (several events), creating a sub-stream
  inside the global stream. The sub-stream must be inserted in-place in the stream: the whole sub-stream should be
  send in-order to the node's children before processing the next incoming event. For example, in Figure \ref{fig:substream},
  the processing of an input event 1 produces a sub-stream of out events 1-1, 1-2 and 1-3. Even if another input event 2
  arrives, it should not be processed before the whole sub-stream 1-1, 1-2 and 1-3 has been produced and sent to the processor's 
  children. This function is called \verb|process|.
  \item Side-effect with exactly-once semantics: The second action possible is to perform a side-effect upon each of the event
  of the sub-stream generated by the \verb|process| method. This side-effect can for example consist in updating a database representing
  a derived view on the data. This method, called \verb|performSideEffect|, must have an exactly-once semantic even in case of failures, so that the user
  can safely define non-idempotent side-effects.
\end{itemize}

\begin{figure}[h]
  \begin{center} 
    \makebox[\textwidth]{\includegraphics[width=1.0\textwidth]{img/stream_processor.png}}
    \caption{A stream processor}
    \label{fig:streamprocessor}
  \end{center}
\end{figure}

\begin{figure}[h]
  \begin{center} 
    \makebox[\textwidth]{\includegraphics[width=1.0\textwidth]{img/substream.png}}
    \caption{In-order insertion of a sub-stream in a stream}
    \label{fig:substream}
  \end{center}
\end{figure}

Another important functional requirement for processors is that the \verb|process| and \verb|performSideEffect| methods can ensure the
sequentiality of asynchronous non-blocking operations (for example, a side-effect or a processing can be done via an asynchronous call to a 
database, but despite the asynchronous nature of the call the processor must wait that the asynchronous call has returned
before processing the next event). Even sub-stream production can be asynchronous, meaning that the production of a sub-stream
can be a composition of asynchronous operations (like pulling from a database with an asynchronous non-blocking driver).

\section{Non-functional requirements}

\subsection{Data Integration}

The Data Integration part must be able to scale up easily. Scale up means that the puller should automatically make the best
use possible of all cores available on a machine in order to parallelize the various pulls. The different parts of the puller 
should also be easily distributable in case of the load if too big for one machine to handle. 

The puller should also be fault-tolerant, meaning that a failure and restart of the system should ensure that no event
is duplicated or lost.

Moreover, the nature of the puller implies that it will spend the majority of its time doing IO to query different data sources.
Those IOs can have various durations depending on the size of the data to pull, the latency and bandwidth of the data sources, etc.
We want to optimize the use of resource (CPU, RAM) despite the fact that the platform is very IO-oriented. Chapter \ref{chap:study} will show how asynchronous non-blocking IO meets these expectations.

Another non-functional requirement is to have clean and composable code source despite its asynchronous nature. Asynchronous
code can indeed lead to maintenance nightmare if the wrong abstractions are used. Chapter \ref{chap:study} will
show that the use of functional programming solves these problems.

\subsection{Journal and Stream Processing}

The Journal and Stream Processing part requires complex non-functional requirements in order to optimize resource consumption and maximize performance.

A common problem with stream processing is to manage the flow rate. A producer can indeed produce events at a rate superior to
the processing rate of a consumer. This problem is even more important when there is
a tree-like structure of stream processors instead of a linear structure. Indeed, the platform should handle the fact that even if sibling processors in the tree
have different processing speeds, they do not block each other based on the slowest sibling. In other words, sibling processors should be
totally decoupled so that when a new event is sent from a parent to its children, the parent does not have to wait that its slowest
children has finished to process the event in order to send the next event to them. This problem should be handled while minimizing
RAM consumption.

Another requirement is no message loss or duplication even in case of a temporary failure of a processor. This means that a processor
that had a transient failure must be able to \textit{replay} the stream from where it was before its crash.

Stream processors should also be easily distributable in order to deal with event flows that are too big
for one machine to handle.



\chapter{Study of functional programming abstractions for concurrency and asynchronicity}

The architecture of the platform is heavily based on functional programming concepts to handle concurrency and
asynchronicity in an effective and composable way. The following section describes and compares these abstractions.


\section{Monadic Futures}

\subsection{The problems of blocking IO with threads}

To handle concurrency and IO, traditional languages use native threads and blocking IO. A thread is a unit of execution that is has its own stack
on the underlining OS, and concurrency is achieved by switching threads on the machine cores. For example, with the blocking IO model,
a thread that is waiting for IO is preempted by the OS. Traditionally threads has a high resource cost, both fixed cost (the default stack size for a thread is 
1 MB on a 64 bit JVM), high context switching cost and high creation time. In case of Web-oriented application, a new thread is generally spawn for each new client,
and if the Web application needs to call several backend services (that is usually the case in modern Service Oriented Architecture), 
this thread will be blocked, doing nothing but using stack space and causing context switching.
Such a model has been proved to be inefficient for a large number of concurrent clients for Web applications that calls various backend services
and/or perform stream-oriented connections as highlighted by James Roper \footfullcite{bib:asyncio}. This is ever more
important when backend services can occasionally be slow / fail. In case of a blocking IO model, clients' threads that requests
this failed service will wait for this service (before a timeout), causing a very high number of threads in the server. As James Roper
stated, this high number of threads prevents the other requests (calling another non-failed service) to be performed efficiently,
because the server spends a lots of its time doing context switching between blocked threads that are doing nothing. This is ever worst if
you have a maximum number of threads allowed in the server (that is usually the case in cloud platforms): new clients can not connect at all
to your server because there is no thread to allow to them. Non-blocking IO servers are also known as evented servers.
Mark McGranaghan highlights this in his article about Threaded vs Evented Servers \footfullcite{bib:threadevent}. If we define \verb|c| the CPU time
that each request takes and \verb|w| the total time of the request including waiting time calling external services, an evented server
performs way better than a threaded server when the ratio \verb|w/c| is high (so when most time of a request is spent waiting for
external services).

\subsection{The problems of asynchronous non-blocking IO with callbacks}
In order to avoid the problems caused by blocking IO, one can use a non-blocking IO model: when a thread is doing an IO operation, it doesn't wait
until the IO is finished but rather provide a mechanism to notify the caller when the IO is finished. Meanwhile, the thread can be used for other tasks, like
serving other web clients. 

The problem is that this kind of asynchronous non-blocking programming can easily lead to hard code maintenance if no proper abstraction is used. The common way
of many languages to deal with asynchronicity is to provide a callback mechanism (Javascript may be the language that uses them the most). A callback is
way to perform an asynchronous operation by providing a function as a parameter of the function that do the asynchronous operation. The parameter function
will be \textit{call back} when the asynchronous operation is finished. An example of a GET HTTP request to a web service in Javascript is shown in Listing 
\ref{lst:jscb}.

\begin{listing}[h]
\begin{minted}[fontsize=\small, frame=lines, framesep=2mm]{javascript}
performHttpGet("http://www.example.com", function(error, response) {
  if (!error) {
    console.log("Response status: " + response.status)
  }
});
\end{minted}
\caption{A callback in Javascript}
\label{lst:jscb}
\end{listing}

Here \verb|function(error, response) {...}| is the user-defined function that is call back when the asynchronous GET request
returned. We see that callbacks are only about side-effects: no value is returned by the \verb|performHttpGet| function.
This causes a serious lack of composability, popularly known as "callback hell". Listing \ref{lst:jspd} shows how to perform several asynchronous operations
sequentially.

\begin{listing}[h]
\begin{minted}[fontsize=\small, frame=lines, framesep=2mm]{javascript}
action1(function(error, response1) {
  if (!error) {
    action2(function(error, response2) {
      if (!error) {
        action3(function(error, response3) {
          if (!error) {
            action4(function(error, response4) {
              // do a side-effect with response4       
            });
          }
        });
      }
    });
  }
});
\end{minted}
\caption{The "pyramid of doom" in Javascript}
\label{lst:jspd}
\end{listing}

Such call is called "pyramid of doom" because the code invariably shifts to the right, and the intermediate steps can not be reused to compose
them later with other operations. 

Moreover, doing concurrent operations with the callback model is not easy also. We want here to perform 2 asynchronous operations in parallel,
and do something with the result (composed of the result of the 2 operations). Listing \ref{lst:jsparr} shows how to do such in standard Javascript.

\begin{listing}[h]
\begin{minted}[fontsize=\small, frame=lines, framesep=2mm]{javascript}
  var results = [];

  function doSomethingWithResults(results) {
    // final callback
  }

  action1(function(error, response) {
      results.push(response);
      if (results.length == 2) {
        doSomethingWithResults(results)
      }
    }
  });

  action2(function(error, response) {
    results.push(response);
    if (results.length == 2) {
      doSomethingWithResults(results)
    }
  });
\end{minted}
\caption{Performing two asynchronous operations concurrently in Javascript}
\label{lst:jsparr}
\end{listing}

The fact that the callback model is based on a closures that performs side-effect prevents easy composability. What I mean by composabilty
is the fact of defining independently various asynchronous operations, and then compose them (sequentially, in parallel) to obtain a composed result
of these actions. Moreover, error handling must be done manually for each asynchronous operation. Monadic futures are an abstraction coming from
functional programming that solves these problems.


\subsection{Monadic futures for composable asynchronous non-blocking programming}

A Future is a monadic abstraction that stands for a value that may be available in the future. Using Scala's notation, a future is a type that is 
parametrized by the type of the value that will eventually be available. For example, \verb|Future[Int]| is a type that represents an eventual integer.
With futures, asynchronous functions return a \verb|Future[ResponseType]| instead of taking a callback function as a parameter. Listing \ref{lst:futures}
shows simple future creations.

\begin{listing}[h]
\begin{minted}[fontsize=\small, frame=lines, framesep=2mm]{scala}
val futureResponse: Future[HttpResponse] = performHttpGet("http://www.example.com")
val futureComputation: Future[Int] = future {
  // do long computation
}
\end{minted}
\caption{Futures in Scala}
\label{lst:futures}
\end{listing}

We see in Listing \ref{lst:futures} that futures can be used for non-blocking IO, but also as an abstraction for concurrency. In the example, the main thread
executing the code does not block on both methods. The "long computation" will be done in another thread as it is encapsulated by a \verb|future {}|.

Behind the scene, futures are multiplexed into a thread pool named a ExecutionContext in Scala. ExecutionContexts can be passed to methods that returned a future.
This allows to decouple the \textit{concurrency semantic} (which tasks should be run concurrently) from the \textit{concurrency implementation} (an ExecutionContext
can for example limit the number of threads it can use, etc.). Twitter's engineer and researcher Marius Eriksen highlights this idea in his 
"Your Server as a Function" paper \ref{bib:serverfunc} where he states that futures are a declarative data-oriented way of doing asynchronous programming.
\\

The term \textit{monadic} comes from Monads, a key abstraction in typed functional programming coming from the Haskell world. Thoroughly defining what a monad
is out of the scope of this thesis, but in a few words a monad is a type that encapsulates another type in order to perform operations on it. Some operations
are mandatory to define a monad. Listing \ref{lst:monad} define the trait in Scala to define a monad, coming from the book Functional Programming in Scala
\footfullcite{bib:fpscala}. 

\begin{listing}[h]
\begin{minted}[fontsize=\small, frame=lines, framesep=2mm]{scala}
trait Monad[F[_]] extends Functor[F] {
  def unit[A](a: => A): F[A]
  def flatMap[A,B](ma: F[A])(f: A => F[B]): F[B]

  def map[A,B](ma: F[A])(f: A => B): F[B] = flatMap(ma)(a => unit(f(a)))
}
\end{minted}
\caption{The Monad trait in Scala}
\label{lst:monad}
\end{listing}

\verb|unit| allows to construct a monad that encapsulate a value of type A, \verb|map| allows to apply a function to the encapsulated value,
and \verb|flatMap| allows to apply a function to the encapsulated value that returns itself a monad.

A Future is a monadic type, meaning that it extends the monad trait and implement the \verb|unit| and \verb|flatMap| methods.
These methods allows powerful compositions between different Futures instances. 

TODO: compare to callback

TODO: solve asynchronous + concurrency ! (PROBLEM: no state (increment a counter...))

\section{Iteratees}



\section{Actor model}

\subsection{The problems of synchronization with threads}

TODO: compare to locks...

TODO: problem: no sequentiality of async op

\chapter{Data Integration}

\section{Functional and non-functional requirements}

The Data Integration part of the platform needs to integrate several data sources to the Journal. Integration means that it must be able 
to detect the changes made to the data, and push events that can be either create event or update event or delete event.
In the following, we call a data entry a \textit{resource}. A resource is a keyed data defined by its id (for example \verb|/client/1| for a
resource of type client of id 1). Each type of resource has a defined set of fields (for example a client
will have a field name, address, ...).

\subsection{Integration of non-evented REST data source}
The data sources that must be integrated exposes REST API. The problem with REST API is that they are not evented, i.e
they are pulled-based and not pushed-based. One must sent an HTTP request to query new data each time they need to.
There exists some techniques to stream data via HTTP 1.1 and the Chunked Transfer Encoding, but the REST APIs that the platform
needs to integrate does not exposes such stream interface. 

Thus, the architecture of this part needs to provide a way to perform incremental 





\chapter{Architecture and implementation of the Journal and Stream Processing part}

\section{Architecture}

As presented in the Requirements chapter, the Journal needs to store the events an append-only list. The Journal should also provide a way for stream processors to subscribe 
to the real-time stream of events. Stream processors can subscribe to themselves in a Tree-like structure (see Figure \ref{fig:tree}), and upon the reception of an event can create a substream of events and perform a side-effect.
\\

\subsection{Naive push-only solutions}

A important problem that will model the architecture is the fact that the Journal can have an event stream rate that is superior to its subscribers. More generally,
any parent node (Journal or processor) can have a output stream rate that is superior to one or several of its children. 
Several simple \textit{push} solutions can be applied, but none of them were applicable for our system.

First, any child can just have a in-memory buffer that store the incoming events not yet processed. However, an in-memory buffer has obvious limitations like the danger
of causing an OutOfMemory exception. Moreover, as said in the non functional requirements part, one goal is to limit the RAM consumption of the platform. Thus, this solution is
not applicable.

Another solution can be for a parent node to wait that all its children have processed the current event to send the next one via an ACK mechanism. 
An obvious issue of this approach is that
the slowest child of a parent will slow down the event stream rate for every of its siblings. This is clearly not acceptable for a scalable system with loose coupling 
between components. Moreover, such a solution implies that the failure of a child stops the stream for its siblings, which is clearly unacceptable.

As a reminder, no message loss is a requirements for the platform, so dropping the events if the stream rate is too fast is not an option.
\\

\subsection{Pull-based streaming with event-sourced processors}

One suitable solution is to use a pull-model instead of a push-model. For each received event, a processor processes it to produce a substream, optionally do some
side effects on the substream, and stores this substream in a local journal. Thus, each processor maintains its own event journal, so each processor is \textit{event-sourced}.
This allows each child to maintain a cursor on the journal of its parent pointing on the next event to pull. Thus, children can have totally different pulling and processing speed, they are not coupled to each other. This approach of pull-based stream system with decoupled multi-consumers using cursors comes from Apache Kafka, a distributed messaging system for log processing created at LinkedIn \footfullcite{bib:kafka}. 

A common problem with pull based-system is the polling part. How do children to known when the next event is ready to be pulled? The naive way to do this is to check every X seconds / milliseconds if the parent has a new event in its journal. This can waste a lot of resources when a parent has no new event for a while. The solution brought by Kafka is to 
perform long-polling: when the child's cursor go the next event, either it pulls the next event if it exists or it blocks until a new event comes in to pull it. Thus, children does not have to pull periodically to know if there is a new event.

However, the term "blocks" does not really get along well with our reactive non-blocking architecture. We therefore have to find a way to implement long-polling without blocking
threads. To do that, we will use a new functional programming abstraction that comes with Futures: Promises. Mixing Futures, Iteratees and Promises, we will be able to implement
an asynchronous non-blocking long-polling system (more details in the Implementation section \ref{sec:streamimplementation}).

Both the Journal and the local journals of processors are persistent using MongoDB. MongoDB is a document-oriented NoSQL database \footfullcite{bib:mongodb} that stores BSON documents (a binary representation of JSON). The format of stored event is a BSON-serialized version of ZEvents (see Listing \ref{lst:zevent}). It contains an id, the name of the resource (for example \verb|/resourceType1/id4|), the user that inserted the event in the Journal, the insertion date, the type of event and the body (data) of the event in a JSON object. To model a journal, we just use a MongoDB collection where we only insert new documents (events). To keep the insertion order of events, an id of type PathId is serialized
into the document. MongoDB provides a built-in id generation mechanism that keep the insertion order, but the fact of ensuring message ordering of substream across processors implies to create a more sophisticated id generation mechanism. This will be explained thoroughly in section \ref{sec:substreamproblem}.
\\

Thus, each event produced by a processor goes into its MongoDB local journal, and children pull events (one by one or by bulk) according to their position in their parent
local journal. If one or several of them is "up-to-date" with the last event of its parent, a long-polling mechanism allows to prevent them to waste resources periodically pulling their parent.

However, this mechanism can be improved. For example, if a parent processor knows that one of its children is "up-to-date", it can directly send him the next event without
passing by the persistent storage to improve efficiency. This approach is described in the next section.

\subsection{Fault-tolerant persistent processors with exactly-once side-effect semantic using Path Ids}
\label{sec:substreamproblem}

Of course, a persistent storage on MongoDB allows fault-tolerance in case of a processor crashes and restarts. When a processor restarts, it checks in MongoDB what was the
last id of the event it successfully processed before crashing, and ask its parent for the next event after this id (it \textit{replays} the stream from where it crashed).
However, if a processor crashes \textit{during} the processing of an event, how to know if it has successfully and entirely processed this event? What it means to process "entirely" an incoming event?
\\

First, we differentiate the \textbf{process} method and the \textbf{performAtomicSideEffect} method in a processor. As stated in the Requirement chapter, the process method 
takes one event and produces a substream of events from it. Its signature is \verb|process(event: I): Enumerator[O]| where I is the type of input events and O the type
of out events. As we saw in the previous parts, an Enumerator is a functional abstraction for a non-blocking stream producer. The function implementing the process interface
should be pure, i.e should not have any side-effect. More precisely, it can have side-effects, but the call semantic of this function is \textit{at-least-once} for each 
event, meaning that the same side-effect can be called several times. Thus, it is ok to have idempotent side-effects. However, for side-effects that are non-idempotent and
thus requires \textit{exactly-once} semantic, one can use the performAtomicSideEffect method whose signature is \verb|performAtomicSideEffect(event: O): Future[Unit]|.
The function is called for each output event of the created sub-stream and is guaranteed to be called exactly-once for each output event even in cases of failures. The
Future[Unit] returns type is a Future returning nothing (normal for a side-effect). The difference of returning Future[Unit] instead of simply returning Unit is that a side-effect
can be asynchronous but returning a Future of nothing allows us to known when this side-effect is finished. This enables to ensure sequentiality of side-effect (the side-effect
of the out event 1 in the substream will be finished when the side-effect of out event 2 begins).

With these two functions, fault-tolerance is handled in a clear manner. process is an user-defined function that is the logic of the processor: how to derive one event to a substream of derived events. As a pure function, it is not a problem if it is called several times for the same input. performAtomicSideEffect is then used on each event to
save the event to a local journal. This operation must me \textit{atomic}. Thus, if a processor crashes, when it recovers it just take the last out event id processed LastID and its parent for the event that generated the substream containing this out event. The parent re-sends this event which is put again in the process method (at-least-once call semantic).
This process method re-creates the substream, and a filter is put on the substream to take only the out events that are \textit{after} the original LastID ouptut event. This
filtered substream is then given to the performAtomicSideEffect method that goes on saving events as usual.
Thus, we have ensured no message loss, no message duplication and guaranteed message ordering in cases of processor crashes. It is easy to see why the "insert in journal" operation in performAtomicSideEffect needs to be atomic: if it is not (say it has 2 steps), if the processor fails between step 1 and step 2, we are in a position where
we don't know if the current out event has been inserted or not where the processor recover. 

This atomicity leads to a potential problem: if we already use the performAtomicSideEffect for inserting output events in the journal, there is no room left for an arbitrary side-effect that can be defined by the user of the library. This is why we define two kinds of processors: \textit{persistent} processors where the performAtomicSideEffect is already implemented to insert output events in a local journal (the type of processor described in this section), and \textit{side-effect} processors that allows to define
an arbitrary atomic side-effect. Side-effect processors will be described in details in section \ref{sec:sideeffectproc}.

Thus, for a persistent processor, the fact that an input event has been "entirely" processed means that all of the events in the substream it has generated are inserted in the local journal.
\\

One key operation that has not been explained is how to know which input event has generated a particular output event generated in a substream. This is a complex task for which the notion of Path Ids has been created.

\subsubsection{Auto generation of tree-like Ids: Path Ids}

The generation of events across the tree of processors can be seen itself as a event tree. For example, when event 1 comes from the Journal and enter in the first
processor, event 1 can generate a substream of events (say event 1-1, event 1-2 and event 1-3). Then, a child processor that generate a 2-event substream
will generate event 1-1-1, event 1-1-2, event 1-2-1, event 1-2-2, event 1-3-1 and event 1-3-2 (see Figure \ref{fig:treepathid}).

\begin{figure}[h]
  \begin{center} 
    \makebox[\textwidth]{\includegraphics[width=1.0\textwidth]{img/id_tree.png}}
    \caption{A generational tree of PathIds}
    \label{fig:treepathid}
  \end{center}
\end{figure}

We see that the generation of events has a tree-shape, and each event can be characterized by a path in this tree (for example, 1-2-1). This idea of PathId is to
auto-generate these path ids when event passes through processors so that it is possible to re-climb the tree given a particular event.

Concerning fault-tolerance, a child processor that recovers after a crash can check its last PathId, remove the last node of it and send the new PathId to its parent. Then, the parent send him the event that it re-processed (re-creating the substream), and thanks to the number of the last node the child can know from where in the substream it crashed. The idea of id generation to be able to retrieve from a child event the parent event comes from Apache Storm, a distributed and fault-tolerant real-time computation framework \footfullcite{bib:storm}.

The implementation part will explains in more details of these PathIds are serialized and deserialized in MongoDB.

The ability of re-climbing the generational tree is also very useful for side-effect stream processors described in next section.

\subsection{Side-effect stream processors}
\label{sec:sideeffectproc}

Persistent processors use their performAtomicSideEffect method to insert the generated events in their local journal. We saw that a persistent processor can only do one
atomic side-effect per generated event in order to guarantee exactly-once side-effect semantic even in cases of failure. So it is not possible to update
an incremental custom view on arbitrary data contained in the event for example.

Thus, we define another type of processor where one can implement an arbitrary side-effect in the performAtomicSideEffect method: side-effect processor. The contract
for this side-effect is that it is atomic if one wants to have the exactly-once side-effect semantic, and moreover the side-effect of event of PathId N should store
this PathId somewhere to be able to retrieve it when it recovers. Note that this is not the same than inserting the event in a journal as in persistent processors:
here we only need to store the PathId, not the entire event. Upon a recover, a method getLastProcessedEventId of signature 
\verb|getLastProcessedEventId(): Future[PathId]| is able to retrieve the last PathId processed to initiate the recover mechanism explained in the previous section.
\\

Thus, side-effect processors does not store the whole list of events but only the last PathId processed. It brings a problem if a child of a side-effect processor
wants to replay past events because it is slow or because it crashed. To solve this, side-effect processors ask their own parent to replay the stream of events.
This is a recursive call until the nearest parent processor that is persistent which is the tree root (Journal) in the worst case. 

Moreover, each intermediate side-effect processor in this recursive call must take care of sending only the minimal amount of messages in substreams. Indeed,
as we re-climb the generational PathId tree, each event send by a parent is put in the process method that generates the substream. We must generally just send
events from this substream from a certain event because all events generated from previous events of the substream has already been processed by children. 

Figure \ref{fig:sideeffectreclimb} illustrates this mechanism. If side-effect processor 2 fails just after the processing of event 1-2-1, TODO
\\

In the end, side-effect processors allows to define an arbitrary side-effect (with some constraints), allows to save disk space because it does not maintain
a local journal, but it implies that event streams replays take longer because a side-effect processor cannot replay the stream itself, it has to ask to its parent.

TODO: side-effect needs to save last process id
TODO: no need to save all events, but replayabilty takes longer

=> at least once for processing, different than side-effect !


\subsection{Adaptive push-pull streaming with back-pressure}

Asynchronous in - out in processors

The pull-based mechanism can be improved to limit the accesses to the local persistent journal.


Optimization: push mode with child states in parent node

explain back-pressure thoroughly, TCP included ! (maybe in its own subsubsection) (see Evernote for bib)

The "ACK" mechanism is at TCP-level in distributed mode -> very efficient !


\subsection{The Journal case: ensuring ordering with non-replayable multiples sources}

Mongo IDS ?

Only in impl part ? no...



\section{Implementation}
\label{sec:streamimplementation}

\subsection{Abstractions choice}

TODO: pas Actors car:
- pas back-pressure auto
- sequentiality of async op pas auto, besoin de stash, in-memory inefficient car remet dans la mailbox a chaque fois...

--> Custom Stream Processor abstraction on top of Iteratees/Futures/Promises !

\subsection{...}

TODO: explains pathid SerDe in Mongo
TODO: explains long-polling promises...


\section{Implementation}
\label{sec:streamimplementation}

\subsection{Abstractions choice}

As stated in the Architecture part, a stream processor is composed of one Iteratee in input of N Enumerators in output (one per child). Distribution is done using HTTP stream on top of TCP. Custom processors on top of Iteratees have been selected over actors for several reasons.

First, actors does not handle back-pressure in a built-in way. But back-pressure is very important for our system in order to optimize the resources used. Back-pressure is even
used from a parent's local journal to its children using the reactive MongoDB driver ReactiveMongo \footfullcite{bib:reactivemongo} that exposes methods returning Iteratee / Enumerators. It is really
convenient to have only one composable abstraction to compose streams with back-pressure from MongoDB or from other processors. 

Moreover, actors does not have a simple mechanism to sequentially compose asynchronous operations. When an actor processes a message and call an asynchronous function (which
returns a Future), it handle the next message in its mailbox meanwhile. An actor cannot "block in a non-blocking way" over asynchronous operations as Iteratees can. There exists
a solution using the Stash trait \footfullcite{bib:stashtrait} that allows to put in local memory the messages that we want to process after the current event has been asynchronously processed, but it is not fault-tolerant if the actor fails and not efficient as messages are swapped between the mailbox and the actor local memory (and not compatible with back-pressure).

In the end, Iteratees, Futures and Promises are better abstractions than actors to handle this problem.

\subsection{PathId serialization and deserialization into MongoDB}

The PathId Scala class is composed of the MongoDB id of the root event inserted in the Journal, plus a Vector of Int that represents the Path in the generational event tree.
A MongoDB id is an id created by the Scala MongoDB driver that is a 24-char hexadecimal string made with the current time plus a local incremental counter in order to guarantee that
each document id is unique and that the id of events are strictly ascendant to retain the order of insertion.

By default, MongoDB uses the "_id" field of a document to put an unique id. Moreover, all MongoDB's collections have a default index on this field. Thus,
for simplicity and efficiency, we will serialize PathId to a hexadecimal string to put as a value of this "_id" field. This serialization has to maintain the order
of events in a local journal. Indeed, ReactiveMongo's stream capabilities from MongoDB allows to get from MongoDB a stream of all the documents of a collection since a particular id. Therefore, this id has to be ascendant for all events of a local journal (to define the order, MongoDB does a simple String comparison from left to right).

The simple way to do this is to first put the root event 24-char id, and then put each integer of the path id on a padded 8-char byte. The padding allows easy deserialization.

With this serialization, events that are the same height in the tree have ids with the same number of chars (so in particular, all event ids generated by a processor have the same size). Moreover, locally in each local journal of processor, all events generated from a root event 01 created before root event 02 will have an id smaller than all events
generated from root event 02 (because they have the same size, and the root event id 01 has a higher id than 02 which are at the left of the serialized PathId). Furthermore,
sibling nodes in the event tree have an incremental number according to their creation order, so order is ensured in a substream. Figure \ref{fig:pathidserde} illustrates this serialization and ordering mechanism. With this model, for each input event, a processor can generate sub-events with an additional 4-byte part at the end of the PathId, so
for each event a processor can generate at maximum 2^32 events (around 4 billion events).

\begin{figure}[h]
  \begin{center} 
    \makebox[\textwidth]{\includegraphics[width=1.0\textwidth]{img/pathidserde.png}}
    \caption{PathId serialization ensuring ordering in MongoDB local journals}
    \label{fig:pathidserde}
  \end{center}
\end{figure}

Listing \ref{lst:pathidserde} shows the code of Path Id and its serialization and deserialization.

\begin{listing}[h]
\begin{minted}[fontsize=\codesize, frame=lines, framesep=2mm]{scala}
case class PathId(rootEvent: String, path: Vector[Int])

object PathId {
  import reactivemongo.bson.utils.Converters
  import java.nio.ByteBuffer

  def apply(rootEvent: String): PathId = PathId(rootEvent, Vector.empty)

  def serialize(id: PathId): String = {
    val array = id.path
    val byteBuffer = ByteBuffer.allocate(array.length * 4)
    val intBuffer = byteBuffer.asIntBuffer
    intBuffer.put(array.toArray)
    byteBuffer.flip()

    id.rootEvent + Converters.hex2Str(byteBuffer.array())
  }

  def deserialize(str: String): PathId = {
    val (idStr, pathStr) = str.splitAt(12*2)

    val byteBuffer = ByteBuffer.allocate(4)
    val arrayBytes = Converters.str2Hex(pathStr)
    val path = arrayBytes.grouped(4).toVector map { bytes =>
      byteBuffer.put(bytes)
      byteBuffer.flip()
      val int = byteBuffer.getInt
      byteBuffer.clear()
      int
    }

    PathId(idStr, path)
  }

  val min = PathId("000000000000000000000000")
}
\end{minted}
\caption{PathId serialization and deserialization}
\label{lst:pathidserde}
\end{listing}


\subsection{Stream processors}




TODO: explains long-polling promises...

\subsection{Journal}




\chapter{Evaluation}

\section{Requirement evaluation}

Concerning the functional requirements, the platform meets the needs by allowing to define data pullers (actors) that incrementally pull various data sources and various resource types.
The pulled data is processed sequentially in a simple pipeline that aggregates, cleans and validates the data before inserting events in the Journal.

Then, the Stream Processing part allows to define a tree of stream processors. A processor can react to events sent by its parent by producing a substream of events towards its children. Substream are inserted in-place in the stream, meaning that the whole substream should be sent to children before processing the next input event. A processor can do a side-effect with a guaranteed exactly-once side-effect semantic, allowing to the user of the library to safely define non-idempotent side-effects.
\\

Concerning the non-functional requirements, all the parts of the architecture are built with an asynchronous non-blocking architecture according to the Reactive Manifesto 
\footfullcite{bib:reactiveManifesto} to optimize performance and resource use by being \textit{event-driven, scalable, resilient and responsive}, the four Reactive Traits.

The Data Integration part is built on top of an Actor system to allow easy concurrency and distribution. Moreover, it makes the best use of resources (CPU, Threads) thanks to a non-blocking implementation. Futures and Iteratees are used to model sequential in-order asynchronous stream processing in a simple, composable and maintainable way. A persistence storage
in MongoDB is used to ensure fault-tolerant pullers: there is no event loss or duplication even in case of failure of a puller. The system is easily distributable 
thanks location transparency that is inherent to the actor model.

The Stream Processing part uses a complex adaptive push-pull model with back-pressure to allow decoupled stream processing while optimizing resource consumption. A stream processor abstraction has been created on top of Iteratees, Futures and Promises. The tree of processors guarantees in-order sequential asynchronous stream processing with fault-tolerance: the temporary failure of a processor is guaranteed without message loss or duplication by allowing a processor to replay the stream from its parent (as shown and explained in section 
\ref{sec:archistream}).


\section{Performance evaluation}

\subsection{Latency and resource consumption}

Performance tests have been performed on the Journal and Stream Processing part on a local machine with a 2.2 GHz processor of 8 cores (so a parallelization factor of 8). We will perform tests on the business use case application described in section \ref{sec:usecasebusiness}.
\\

First, we measure the end-to-end latency between the time when an event is inserted and the time when the resulting dashboard update(s) have been entirely performed. Figure \ref{fig:latencyplot} show the plot of the end-to-end latency between the Journal and a Dashboard when we increase the push rate of events inserted in the Journal.

We notice that before a threshold of roughly 170 events inserted in the Journal per second, the latency between the Journal and a Dashboard is constant at 8 ms. This means that the processors lower in the tree structure (snapshot, flatSnapshot and the dashboards) can handle the push rate of the Journal and are not late in the stream (push-mode).
However, after 170 events per second, the latency becomes linear with the push rate. This means that dashboards starts to be late in the stream, and are forced to replay events at their rate because their processing time is too slow. Thus, the resultant plot is either constant (up-to-date child processors) or linear (late child processors), which is an expected and good result for scalability.
\\

\begin{figure}[h]
  \begin{center} 
    \makebox[\textwidth]{\includegraphics[width=1.0\textwidth]{img/plotlatency.png}}
    \caption{End-to-end latency between the Journal and a Dashboard while varying the Journal push rate}
    \label{fig:latencyplot}
  \end{center}
\end{figure}

Furthermore, during this performance test, the resource consumption (JVM Heap space, number of threads) has been profiled. Figure \ref{fig:plotheapspace} and Figure \ref{fig:plotthreads} show the JVM Heap space used and the number of threads used while increasing the event push rate on the Journal. We notice that resource consumption is constant even when we increase a lot the push event rate, which validates the fact that the platform uses a predictable amount of resource that does not increase too much with the load (optimization of resource consumption).

\begin{figure}[h]
  \begin{center} 
    \makebox[\textwidth]{\includegraphics[width=1.0\textwidth]{img/plotheapspace.png}}
    \caption{JVM heap space consumption while varying the Journal push rate}
    \label{fig:plotheapspace}
  \end{center}
\end{figure}

\begin{figure}[h]
  \begin{center} 
    \makebox[\textwidth]{\includegraphics[width=1.0\textwidth]{img/plotthreads.png}}
    \caption{Number of threads used while varying the Journal push rate}
    \label{fig:plotthreads}
  \end{center}
\end{figure}


\subsection{Fault-tolerance}

In this part, we define a performance test to measure the recovery time of a processor that recovers from a crash and must replay 1000 events that happened when it was crashed. Figure \ref{fig:barchart} shows the resultant bar chart.
Using the tree structure of the business use case application presented in section \ref{sec:usecasebusiness}, the Snapshot processor is first killed and then restart. Its parent is the Journal, a persistent processor, so it has only one level to climb in the tree to replay the stream. Then, FlatSnapshot is killed. It is at level 2 in the tree, but its parent is a persistent processor (Snapshot). Therefore, it has also only one level in the tree to climb to replay the stream. As a result, its replay time is roughly the same than Snapshot (the processing time of these two processors is equivalent). Last but not least, a Dashboard is killed. As its parent is a side-effect processor (FlatSnapshot), it has to climb 2 levels to replay the stream (until Snapshot). Moreover, the processing time of a dashboard is slightly superior than other processors. As a result of these two factors (side-effect parent and slightly longer processing time), we see that Dashboard takes more time to replay the 1000 events that it missed, which is expected according to our model.

\begin{figure}[h]
  \begin{center} 
    \makebox[\textwidth]{\includegraphics[width=1.0\textwidth]{img/barchart.png}}
    \caption{Recovery time of processors for a replay of 1000 events}
    \label{fig:barchart}
  \end{center}
\end{figure}




\newpage
\printbibliography

\newpage
\listoffigures

\newpage
\listoflistings

\end{document}
