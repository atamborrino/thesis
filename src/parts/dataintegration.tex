\chapter{Data Integration}

\section{Functional and non-functional requirements}

The Data Integration part of the platform needs to integrate several data sources to the Journal. Integration means that it must be able 
to detect the changes made to the data, and push events that can be either create event or update event or delete event.
In the following, we call a data entry a \textit{resource}. A resource is a keyed data defined by its id (for example \verb|/client/1| for a
resource of type client of id 1). Each type of resource has a defined set of fields (for example a client
will have a field name, address, ...).

\subsection{Integration of non-evented REST data source}
The data sources that must be integrated exposes REST API. The problem with REST API is that they are not evented, i.e
they are pulled-based and not pushed-based. One must sent an HTTP request to query new data each time they need to.
There exists some techniques to stream data via HTTP 1.1 and the Chunked Transfer Encoding, but the REST APIs that the platform
needs to integrate does not exposes such stream interface. 

Thus, the architecture of this part needs to provide a way to perform incremental 




